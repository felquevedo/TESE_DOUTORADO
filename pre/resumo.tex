O projeto e verificação estrutural de túneis profundos envolvem diversos parâmetros geotécnicos bem como a necessidade de prever a convergência (fechamento) e intuir a estabilidade da seção do túnel. Contudo, o campo de deformações e tensões ao redor da cavidade depende de diversos fatores inter-relacionados, tais como, a profundidade, a geometria da seção, a anisotropia das tensões in situ, a heterogeneidade do maciço, o processo construtivo e a interação com o revestimento. Somado a essa complexidade a reologia do maciço é um fator crucial. A rigor o comportamento do maciço é composto por uma parcela instantânea (elástica ou elastoplástica) e diferida (viscosa). O objetivo principal dessa tese é formular, programar e validar, no contexto dos elementos finitos, um modelo constitutivo elastoplástico-viscoplástico geral capaz de reproduzir casos da realidade e permitir estudos sobre esse comportamento. Para a parcela instantânea serão adotados modelos elastoplásticos de Drucker-Prager e Mohr-Coulomb (bem como suas simplificações independentes da pressão, von-Mises e Tresca, respectivamente) com lei endurecimento e amolecimento multinear governada pela variável interna coesiva. Já o comportamento diferido será estudado através de um modelo viscoplástico perfeito conforme a teoria da sobretensão de Perzyna com as mesmas superfícies da elastoplasticidade. A solução em estado plano de deformações, axissimetria e tridimensional será programada em Fortran90. O processo de escavação e colocação do revestimento é simulado através do método de ativação e desativação dos elementos finitos.