O projeto e verificação estrutural de túneis profundos envolvem diversos parâmetros geotécnicos bem como a necessidade de prever a convergência (fechamento) e intuir a estabilidade da seção do túnel. Contudo, o campo de deformações e tensões ao redor da cavidade depende de diversos fatores inter-relacionados, tais como, a profundidade, a geometria da seção, a anisotropia das tensões \textit{in situ}, a heterogeneidade do maciço, o processo construtivo e a interação com o revestimento. Somado a essa complexidade a reologia do maciço é um fator crucial. Um aspecto comum na probelmática de túneis é o comportamento diferido ao longo do tempo. O objetivo principal dessa tese foi formular, programar e validar, no contexto dos elementos finitos, um modelo constitutivo acoplado. Capaz de simular as deformações instantâneas (elásticas ou elastoplásticas) juntamente com o comportamento diferido (viscoso). Para a parcela instantânea foi adotado um modelo elastoplástico de Drucker-Prager com lei de endurecimento/amolecimento governada pela variável interna coesiva. No comportamento diferido foi utilizado um modelo viscoplástico conforme a teoria da sobretensão de Perzyna com a mesma superfície de elastoplasticidade, porém perfeita. Dessa forma, foi desenvolvido um esquema de integração numérica para lidar, de forma geral, com o comportamento acoplado conjuntamente com suas variáveis internas. O modelo foi generalizado para abordar problemas em estado plano de deformações, axissimetria e tridimensional e implementado no \textit{software} ANSYS utilizando o seu recurso programável \textit{Usermat}. O processo de escavação e colocação do revestimento é simulado através da ativação e desativação dos elementos finitos. Por fim é apresentado algumas validações, verificações e análises paramétricas. O modelo acoplado representou XX XX do que apenas o modelo viscoplástico. 