O projeto e verificação estrutural de túneis profundos envolvem diversos parâmetros geotécnicos bem como a necessidade de prever a convergência (fechamento) e a estabilidade da seção do túnel. Contudo, o campo de deformações e tensões ao redor da cavidade depende de diversos fatores inter-relacionados, tais como, a profundidade, a geometria da seção, a anisotropia das tensões \textit{in situ}, a heterogeneidade do maciço, o processo construtivo e a interação com o revestimento. Somado a essa complexidade a reologia do maciço é um fator crucial. Um aspecto comum na problemática de túneis é o comportamento diferido ao longo do tempo. O objetivo principal dessa tese foi formular, programar e validar, no contexto dos elementos finitos em estado plano de deformações, axissimetria e tridimensional, um modelo constitutivo acoplado capaz de simular as deformações instantâneas (elásticas ou elastoplásticas) juntamente com o comportamento diferido (viscoso). Para a parcela instantânea foi adotado um modelo elastoplástico de Drucker-Prager com lei de endurecimento/amolecimento governada pela variável interna coesiva. No comportamento diferido foi utilizado um modelo viscoplástico conforme a teoria da sobretensão de Perzyna com a mesma superfície de elastoplasticidade, porém perfeita. Portanto, é apresentado um esquema de integração numérica para lidar, de forma geral, com o comportamento acoplado conjuntamente com suas variáveis internas. O modelo foi implementado no \textit{software} ANSYS 2021R1 utilizando o seu recurso programável USERMAT. O processo de escavação e colocação do revestimento é simulado através da ativação e desativação dos elementos finitos. Verificações com soluções analíticas e numéricas demonstraram que modelo desenvolvido funcionou adequadamente. Análises paramétricas demonstraram diferenças da ordem de 10\% a 30\% na comparação desse modelo acoplado com o modelo apenas viscoplástico. Também foi estudado a influência desse modelo em seções transversais elípticas demonstrando a sua importância quando ocorrem plastificações no entorno do maciço.