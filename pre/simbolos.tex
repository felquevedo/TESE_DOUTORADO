

\item[\textbf{Latino minúsculo:}]
\item[$c$]				Coesão;
\item[$c_{TR}$]			Coesão na superfície de escoamento de Tresca;
\item[$c_{VM}$]			Coesão na superfície de escoamento de von-Mises;
\item[$c_0$]			Coesão residual no modelo elastoplástico;
\item[$c_1$]			Coesão para o modelo viscoplástico;
\item[$d_0$]			Distância não revestida que acompanha a frente de escavação;
\item[$d_0^*$]			Distância não revestida adimensional que acompanha a frente de escavação;
\item[$ \greenll $] 	Tensor de deformações de Green-Lagrange;
\item[$\euml, \edoisl, \etresl$] Vetores ortonormais da base que define o espaço $\mathbb{R}^3$;
\item[$f$]				Função de escoamento;
\item[$f_0$]			Parâmetro de tensão no modelo viscoplástico;
\item[$\fl$]					Campo de acelerações referente às forças de corpo;
\item[$g$]				Função potencial;
\item[$\gllum, \glldois, \glltres$] Gradientes para determinar o vetor de fluxo;
\item[$\dgdsll$]		Vetor de fluxo;
\item[$\hl$]			Vetor que contém o conjunto dos módulos de endurecimento para as forças associadas;
\item[$h_s$]			Profundidade de \textit{Spalling} no perímetro de uma seção;
\item[$k$] 				Parâmetro da superfície de Drucker-Prager;
\item[$l_0$] 			Dimensão característica para hipótese das pequenas perturbações;
\item[$n$] 				Parâmetro adimensional que dá a forma potencial na lei viscoplástica;
\item[$ p $]			Pressão hidrostática;
\item[$ p_i $]			Pressão interna no interior da seção do túnel;
\item[$ p_{lim} $] 		Pressão limite de plastificação do maciço;
\item[$ p_{eq} $] 		Pressão de equilíbrio de um túnel revestido;
\item[$ p_\infty $]		Pressão geostática-hidrostática;
\item[$ p_\infty^* $]	Pressão geostática-hidrostática adimensional;
\item[$ q $]	Tensão equivalente de von-Mises;
\item[$\ql$] conjunto de forças termodinâmicas (escalares ou tensoriais) associadas às variáveis internas;
\item[$ q_0 $] 			Resistência à compressão uniaxial do maciço;
\item[$ r $]			Coordenada radial no plano da seção transversal do túnel;
\item[$ \sll $]			Tensor de tensões desviadoras;
\item[$ t $] 			Tempo;
\item[$ \ul $] 		    Deslocamento da partícula material;
\item[$ u(r) $] 		Deslocamento radial em uma coordenada radial $r$;
\item[$ u_x $] 			Deslocamento na direção x;
\item[$ u_y $] 			Deslocamento na direção y;
\item[$ u_z $] 			Deslocamento na direção z;
\item[$ u_{z_{max}} $] 	Deslocamento máximo na direção z;
\item[$ u_\varepsilon $] Deslocamento uniforme da seção de um túnel raso;
\item[$ u_\delta $] 	 Distorção da seção de um túnel raso;
\item[$ \Delta u_y $] 	 Translação vertical uniforme da seção de um túnel raso;
\item[$ w $] 	 		Energia interna específica;
\item[$\xl$]			Vetor posição da partícula material;
\item[$x$]				Eixo x (eixo longitudinal do túnel);
\item[$y$]				Eixo y;
\item[$z$]				Eixo z (eixo da profundidade do túnel);

\item[\textbf{Latino maiúsculo:}]
\item[$C_1, C_2, C_3$]	Constantes para determinar o vetor de fluxo;
\item[$D$]				Diâmetro equivalente da seção do túnel;
\item[$\Dllll$] Tensor constitutivo elástico de quarta ordem simétrico;
\item[$\Dllll^{ep}$] Tensor constitutivo elastoplástico contínuo;
\item[$\Dllll^{vp}$] Tensor constitutivo viscoplástico contínuo;
\item[$E$]				Módulo de Young;
\item[$E^*$]			Módulo de Young adimensional;
\item[$E_h$]			Módulo de deformação médio horizontal na parte superior da crosta terrestre;
\item[$\Fll$]			Gradiente da transformação geométrica;
\item[$H$]				Profundidade do túnel;
\item[$I_1,I_2,I_3$] 	Invariantes do tensor de deformações ou do tensor de tensões;
\item[$J_1,J_2,J_3$] 	Invariantes do tensor de tensões desviadoras;
\item[$ J $] 	Jacobiano da transformação;
\item[$K$]				Coeficiente de empuxo do maciço;
\item[$K_r$]			Rigidez equivalente do revestimento;
\item[$K_r^*$]			Rigidez equivalente adimensional do revestimento;
\item[$L_s$]			Comprimento de \textit{Spalling} no perímetro de uma seção;
\item[$ P $] 			Pressão de confinamento em um ensaio triaxial;
\item[$ P_0 $] 			Tensão média no interior do maciço;
\item[$ Q $] 			Pressão axial em um ensaio triaxial;
\item[$ R $]			Raio da seção transversal do túnel;
\item[$ R^p $]			Raio da zona plástica ao redor da seção do túnel;
\item[$ S $] 			Superfície que atua o vetor tensão;
\item[$ S_0 $] 			Tensão desviadora no entorno do maciço;
\item[$ S_T $] 			Região da superfície do domínio com condição de contorno imposta;
\item[$ \Tl $] 			Vetor tensão;
\item[$ \Tl^d $] 		Vetor tensão imposto na fronteira $S_T$;
\item[$ U $] 			Convergência da seção do túnel;
\item[$ U_{0} $] 		Convergência da seção do túnel na cota não revestida a partir da face de escavação;
\item[$ U_{eq} $] 		Convergência de equilíbrio de um túnel revestido;
\item[$ U_{max} $] 		Convergência máxima da seção do túnel;
\item[$ V $] 			Velocidade de avanço do túnel;
\item[$ V^* $] 			Velocidade adimensional de avanço do túnel;
\item[$ V_a $] 			Volume de uma amostra em um ensaio triaxial;
\item[$ V_{bacia} $]	Volume recalcado da bacia de assentamento de um túnel raso;

\item[\textbf{Grego minúsculo:}]
\item[$\alphal$] Conjunto de variáveis internas;
\item[$\beta$] Parâmetro da superfície de Drucker-Prager;
\item[$ \gammal $] 		Campo de acelerações referente às forças inerciais;
\item[$\gamma_m$] 		Peso específico do maciço;
\item[$\varepsilon_1,\varepsilon_2,\varepsilon_3$] 	Deformações principais;
\item[$ \varepsilonll $] 	Tensor de deformações;
\item[$\varepsilonll^e$] Deformação elástica;
\item[$\varepsilonll^p$] Deformação plástica;
\item[$\varepsilonll^{vp}$] Deformação viscoplástica;
\item[$\bar{\varepsilon}^{vp}$] Magnitude da deformação viscoplástica acumulada;
\item[$ \varepsilon_0 $] Deformação a partir da qual tem-se a coesão residual $c_0$;
\item[$ \varepsilon_a $] Deformação axial de uma amostra em um ensaio triaxial;
\item[$ \eta $] 		Constante de viscosidade dinâmica;
\item[$\eta_1,\eta_2,\eta_3$] 	Direções principais;
\item[$ \theta $] 		Ângulo de Lode;
\item[$\lambda$] 	Magnitude da deformação plástica ou viscoplástica;
\item[$\lambda^e$] 	Coeficiente de Lamè;
\item[$\mu^e$] 	Coeficiente de Lamè;
\item[$ \nu $] 			Coeficiente de Poisson;
\item[$ \xi_{H} $] 				Coordenada do espaço de Haigh-Westergaard;
\item[$ \phi $] 		Ângulo de atrito do maciço;
\item[$ \rho $] 		Densidade do domínio;
\item[$ \rho_{H} $] 		Coordenada do espaço de Haigh-Westergaard;

\item[$ \sigma_{oct} $] 	Tensão normal octaédrica;
\item[$ \sigma_{v} $] 	Tensão vertical no interior do maciço;
\item[$ \sigma_{\theta \theta} $] Tensão no maciço na direção ortorradial;
\item[$ \sigma_{zz} $] 	Tensão no maciço ortogonal ao plano da seção do túnel;
\item[$ \sigma_{rr} $] 	Tensão no maciço na direção radial;
\item[$ \sigmall $] 	Tensor de tensões;
\item[$ \sigmall_0 $] 	Tensor de tensões inical;
\item[$ \tau_{oct} $] 	Tensão cisalhante octaédrica;
\item[$\psi$] energia livre específica;

\item[\textbf{Grego maiúsculo:}]
\item[$ \Gamma $]		Conjunto de tensões plasticamente admissíveis;
\item[$ \Sigma $]		Superfície de descontinuidade no interior do domínio;
\item[$ \Phi $]		Função de sobretensão;
\item[$ \Omega $] 		Domínio contínuo;
\item[$ \Omega' $] 		Subdomínio contínuo;

\item[\textbf{Numéricos, operadores e funções matemáticas:}]
\item[$ \Umll $] 	Tensor de segunda ordem unitário;
\item[$\Umllll$] Tensor unitário de quarta ordem;
\item[det($\cdot$)] 	Função determinante;
\item[tr($\cdot$)] 	Função traço;
\item[$ \divl $] 	Operador divergente;
\item[$ \nabla $] 	Operador Gradiente;


















