

\item[\textbf{Latino minúsculo:}]
\item[$c$]				Coesão;
\item[$c_1, c_2, c_3$]	Constantes para determinar o vetor de fluxo;
\item[$c_{TR}$]			Coesão considerando superfície de escoamento de Tresca;
\item[$c_{VM}$]			Coesão considerando superfície de escoamento de von-Mises;
\item[$c_0$]			Coesão residual no modelo elastoplástico;
\item[$c_1$]			Coesão para o modelo viscoplástico;
\item[$d_0$]			Distância não revestida que acompanha a frente de escavação;
\item[$d_0^*$]			Distância não revestida adimensional que acompanha a frente de escavação;
\item[$e_{rev}$]		Espessura do revestimento;
\item[$ \greenll $] 	Tensor de deformações de Green-Lagrange;
\item[$\euml, \edoisl, \etresl$] Vetores ortonormais da base que define o espaço $\mathbb{R}^3$;
\item[$f$]				Função de escoamento;
\item[$f^p$]				Função de escoamento plástica;
\item[$f^{vp}$]				Função de escoamento viscoplástica;
\item[$\dfdsl$]			Vetor de derivadas da função de escoamento com relação as tensões (notação de Voigt);
\item[$\dfdql$]			Vetor de derivadas da função de escoamento com relação as forças associadas (notação de Voigt);
\item[$f_0$]			Parâmetro de tensão no modelo viscoplástico;
\item[$\fl$]					Campo de acelerações referente às forças de corpo;
\item[$g$]				Função potencial;
\item[$g^p$]				Função potencial plástica;
\item[$g^{vp}$]				Função potencial viscoplástica;
\item[$\gllum, \glldois, \glltres$] Gradientes para determinar o vetor de fluxo;
\item[$\gl_1, \gl_2, \gl_3$] Gradientes para determinar o vetor de fluxo (notação de Voigt);
\item[$\dgdsll$]		Vetor de fluxo;
\item[$\dgdsl$]			Vetor de fluxo (notação de Voigt);
\item[$\hl$]			Vetor que contém o conjunto dos módulos de endurecimento para as forças associadas;
\item[$h_s$]			Profundidade do fenômeno de \textit{Spalling} no perímetro do túnel;
\item[$i$] 				Contador de iterações de equilíbrio;
\item[$i_p$]			Ponto de integração;
\item[$i_p$]			Contador do passo de escavação;
\item[$k$] 				Parâmetro da superfície de Drucker-Prager;
\item[$k_{cut}$] 		Indica que é necessário dividir o subpasso pela metade;
\item[$l_0$] 			Dimensão característica para hipótese das pequenas perturbações;
\item[$l_a,~l_b,~l_c$] 			Dimensão do modelo de ensaio triaxial;
\item[$mL_{1}$] 				Razão entre o primeiro e último elemento além da região de refinamento próximo ao túnel;
\item[$mL_{x_1}$] 				Razão entre o primeiro e último elemento na largura do domínio além da região de refinamento próxima ao túnel;
\item[$mL_{y_2}$] 				Razão entre o primeiro e último elemento na altura do domínio além da região de refinamento próxima ao túnel;
\item[$mL_{z_2}$] 				Razão entre o primeiro e último elemento no trecho não escavado do modelo 3D;
\item[$mR_{1}$] 				Razão entre o primeiro e último elemento na região de refinamento próximo ao túnel;
\item[$n$] 				Parâmetro adimensional que dá a forma potencial na lei viscoplástica;
\item[$n$] 				Contador de subpassos;
\item[$n_e$] 				Número de elementos;
\item[$nL_{1}$] 				Número de elementos além da região de refinamento próximo ao túnel do domínio 3D;
\item[$nL_{x_1}$] 				Número de elementos na largura do domínio além da região de refinamento próxima ao túnel;
\item[$nL_{y_2}$] 				Número de elementos na altura do domínio além da região de refinamento próxima ao túnel;
\item[$nL_{z_2}$] 				Número de elementos no trecho não escavado do domínio 3D;
\item[$n_p$] 				Número de pontos de integração;
\item[$n_p$] 				Número de passos de escavação;
\item[$n_{p_1}$] 				Número de passos na primeira escavação;
\item[$nR_{i}$] 				Número de elementos na direção ortoradial ao longo de $Ri$;
\item[$n_{rev}$] 				Número de elementos na espessura do revestimento;
\item[$nr_{max}$] 				Limite de iterações de Newton-Raphson locais no algoritmo elastoplástico;
\item[$nR_{1}$] 				Número de elementos na região de refinamento próximo ao túnel;
\item[$nR_{v}$] 				Número de elementos na altura da seção escavada;
\item[$nR_{h}$] 				Número de elementos na largura da seção escavada;
\item[$n_{eqit}$] 				Limite de iterações de equilíbrio por subpasso;
\item[$ p $]			Pressão hidrostática;
\item[$ p_{eq} $] 		Pressão de equilíbrio de um túnel revestido;
\item[$ p_i $]			Pressão interna no interior da seção do túnel;
\item[$ p_{lim} $] 		Pressão limite de plastificação do maciço;
\item[$ p_{v} $] 		Pressão vertical no domínio;
\item[$ p_{h} $] 		Pressão horizontal no domínio;
\item[$ p_\infty $]		Pressão geostática-hidrostática;
\item[$ p_\infty^* $]	Pressão geostática-hidrostática adimensional;
\item[$ q $]	Tensão equivalente de von-Mises;
\item[$\ql$] conjunto de forças termodinâmicas (escalares ou tensoriais) associadas às variáveis internas;
\item[$ q_0 $] 			Resistência à compressão uniaxial do maciço;
\item[$ r $]			Coordenada radial no plano da seção transversal do túnel;
\item[$ \sll $]			Tensor de tensões desviadoras;
\item[$ t $] 			Tempo;
\item[$ \Delta t_{min} $] 	 incremento de tempo limite para precisão do algoritmo viscoplástico;
\item[$ t_p $] 			Tempo do passo;
\item[$ \ul $] 		    Deslocamento da partícula material;
\item[$ \ul $] 		    Vetor deslocamentos nodais;
\item[$ \ul_e $] 		    Vetor de deslocamentos nodais do elemento;
\item[$ u(r) $] 		Deslocamento radial em uma coordenada radial $r$;
\item[$ u_x $] 			Deslocamento na direção x;
\item[$ u_y $] 			Deslocamento na direção y;
\item[$ u_z $] 			Deslocamento na direção z;
\item[$ u_{z_{max}} $] 	Deslocamento máximo na direção z;
\item[$ u_\varepsilon $] Deslocamento uniforme da seção de um túnel raso;
\item[$ u_\delta $] 	 Distorção da seção de um túnel raso;
\item[$ \Delta u_y $] 	 Translação vertical uniforme da seção de um túnel raso;
\item[$ w $] 	 		Energia interna específica;
\item[$\xl$]			Vetor posição da partícula material;
\item[$x$]				Eixo x;
\item[$y$]				Eixo y;
\item[$y_f$]			Cota no eixo y da face de escavação no domínio AXI;
\item[$y_r$]			Cota no eixo y da face do revestimento no domínio AXI;
\item[$z$]				Eixo z;
\item[$z_f$]			Cota no eixo z da face de escavação no domínio 3D;
\item[$z_r$]			Cota no eixo z da face do revestimento no domínio 3D;

\item[\textbf{Latino maiúsculo:}]
\item[$\All$]	Matriz algorítmica do esquema de integração;
\item[$\Bll$]	Matriz que relaciona os deslocamentos nodais com as deformações no interior do elemento;
\item[$D$]				Diâmetro equivalente da seção do túnel;
\item[$\Dllll$] Tensor constitutivo elástico de quarta ordem simétrico;
\item[$\Dllll^{ep}$] Tensor constitutivo elastoplástico contínuo;
\item[$\Dllll^{vp}$] Tensor constitutivo viscoplástico contínuo;
\item[$\Dll$] Tensor constitutivo (notação de Voigt);
\item[$\Dll^{alg}$] Tensor constitutivo consistente com o esquema de integração (notação de Voigt);
\item[$\Dll^{e}$] Tensor constitutivo elástico (notação de Voigt);
\item[$\Dll^{ep}$] Tensor constitutivo elastoplástico contínuo (notação de Voigt);
\item[$\Dll^{vp}$] Tensor constitutivo viscoplástico contínuo (notação de Voigt);
\item[$E$]				Módulo de Young;
\item[$E^*$]			Módulo de Young adimensional;
\item[$E_h$]			Módulo de deformação médio horizontal na parte superior da crosta terrestre;
\item[$E_{rev}$]			Módulo de Young do revestimento;
\item[$\Fll$]			Gradiente da transformação geométrica;
\item[$\Fl_{int}$]			Vetor de forças internas globais;
\item[$\Fl_{int_e}$]			Vetor de forças internas do elemento;
\item[$\Fl_{ext}$]			Vetor de forças externas globais;
\item[$\Fl_{ext_e}$]			Vetor de forças externas do elemento;
\item[$\Fl_{\varepsilon_{0_e}}$]	Vetor de forças internas do elemento devido às deformações iniciais;
\item[$\Fl_{\sigma_{0_e}}$]	Vetor de forças internas do elemento devido às tensões iniciais;
\item[$\Fl_{V_e}$]	Vetor de forças de volume do elemento;
\item[$\Fl_{S_e}$]	Vetor de forças de superfície no contorno livre do elemento;
\item[$\Fl_{C_e}$]	Vetor de forças de contato entre elementos vizinhos;
\item[$\Fl_{N_e}$]	Vetor de forças nodais do elemento;

\item[$\Fl_{\varepsilon_{0}}$]	Vetor de força interna global devido às deformações iniciais;
\item[$\Fl_{\sigma_{0}}$]	Vetor de força interna global devido às tensões iniciais;
\item[$\Fl_{V}$]	Vetor de força de volume global;
\item[$\Fl_{S}$]	Vetor de força de superfície no contorno livre global;
\item[$\Fl_{N}$]	Vetor de força nodal global;

\item[$\Fl_{p}$]	Vetor de forças do passo;

\item[$H$]				Profundidade do túnel;
\item[$I_1,I_2,I_3$] 	Invariantes do tensor de deformações ou do tensor de tensões;
\item[$J_2,J_3$] 	Invariantes do tensor de tensões desviadoras;
\item[$J$] 		Jacobiano da transformação;
\item[$\Jll$] 		Jacobiano da transformação;
\item[$J_{i_p}$ ] 				Determinante do Jacobiano da transformação no ponto de integração $i_p$;
\item[$K$]				Coeficiente de empuxo do maciço;
\item[$K_{rev}$]			Rigidez equivalente do revestimento;
\item[$K_{rev}^*$]			Rigidez equivalente adimensional do revestimento;
\item[$\Kll_e$]			Matriz de rigidez do elemento;
\item[$\Kll$]			Matriz de rigidez global;
\item[$L_p$]			Tamanho do passo de escavação;
\item[$L_{p_e}$]		Tamanho do elemento finito no trecho escavado;
\item[$L_s$]			Comprimento de \textit{Spalling} no perímetro de uma seção;
\item[$L_x$]			Largura transversal do domínio EPD e 3D ou raio total do modelo AXI;
\item[$L_{x_1}$]			Largura do domínio além da região de refinamento próxima ao túnel;
\item[$L_y$]			Altura do domínio EPD e comprimento longitudinal do domínio AXI;
\item[$L_{y_1}$]			Altura do modelo EPD e 3D além da região de refinamento próxima ao túnel ou comprimento do trecho escavado no domínio AXI;
\item[$L_{y_2}$]		Comprimento longitudinal do trecho não escavado no domínio AXI;
\item[$L_z$]			Comprimento longitudinal do domínio 3D;
\item[$L_{z_1}$]		Comprimento longitudinal do trecho escavado no domínio 3D;
\item[$L_{z_2}$]		Comprimento longitudinal do trecho não escavado no domínio 3D;
\item[$\Lll$]			Matriz triangular inferior da fatorização de Cholesky;
\item[$\Mll$]			Pré-condicionador do Método dos Gradientes Conjugados Pré-Condicionado;
\item[$\Nll$]			Matriz contendo as funções de interpolação;
\item[$ P $] 			Pressão de confinamento em um ensaio triaxial;
\item[$ P_0 $] 			Tensão média no interior do maciço;
\item[$ Q $] 			Pressão axial em um ensaio triaxial;
\item[$ R $]			Raio da seção transversal do túnel;
\item[$ R_1 $]			Raio da região de refinamento próxima ao túnel;
\item[$R_{i}$]			Raio da interface entre o túnel e o maciço;
\item[$R_{h_i}$]		Raio horizontal na interface entre o túnel e o maciço;
\item[$R_{v_i}$]		Raio vertical na interface entre o túnel e o maciço;
\item[$ R^p $]			Raio da zona plástica ao redor da seção do túnel;
\item[$\Rl_i$]			Vetor resíduo global do método de Newton-Raphson;
\item[$ S $] 			Superfície que atua o vetor tensão;
\item[$ S_0 $] 			Tensão desviadora no entorno do maciço;
\item[$ S_T $] 			Região da superfície do domínio com condição de contorno imposta;
\item[$ \Tl $] 			Vetor tensão;
\item[$ \Tl^d $] 		Vetor tensão imposto na fronteira $S_T$;
\item[$ TOL $] 			Tolerância de convergência do algoritmo de Newton-Raphson local;
\item[$ U $] 			Convergência da seção do túnel;
\item[$ U_i $] 			Convergência em $r = R_i$;
\item[$ U_{0} $] 		Convergência da seção do túnel na cota não revestida a partir da face de escavação;
\item[$ U_{eq} $] 		Convergência de equilíbrio de um túnel revestido;
\item[$ U_{int} $] 		Energia potencial interna de deformação;
\item[$ U_{max} $] 		Convergência máxima da seção do túnel;
\item[$ W_{ext} $] 		Trabalho das forças externas;
\item[$ W_{i_p} $] 		Peso referente ao ponto de integração $i_p$;
\item[$ V $] 			Velocidade de avanço do túnel;
\item[$ V^* $] 			Velocidade adimensional de avanço do túnel;
\item[$ V_a $] 			Volume de uma amostra em um ensaio triaxial;
\item[$ V_{bacia} $]	Volume recalcado da bacia de assentamento de um túnel raso;
\item[$ V_{p} $]	Velocidade do passo de escavação;
\item[$V(\Delta \ul)$]	Potencial do Método dos Gradientes Conjugados;

\item[\textbf{Grego minúsculo:}]
\item[$\alphal$] Conjunto de variáveis internas;
\item[$\beta_1,\beta_2,\beta_3$] Parâmetro da superfície de Drucker-Prager;
\item[$ \gammal $] 		Campo de acelerações referente às forças inerciais;
\item[$\gamma_m$] 		Peso específico do maciço;
\item[$\dvarepsilonll$]		Campo de deformações virtuais;
\item[$\dvarepsilonl$]		Campo de deformações virtuais (notação de Voigt);
\item[$\dul$]		Campo de deslocamentos virtuais;
\item[$\varepsilon_1,\varepsilon_2,\varepsilon_3$] 	Deformações principais;
\item[$ \varepsilonll $] 	Tensor de deformações;
\item[$ \varepsilonll_I $] 	Tensor de deformações iniciais;
\item[$\varepsilonll^e$] Deformação elástica;
\item[$\varepsilonll^p$] Deformação plástica;
\item[$\varepsilonll^{vp}$] Deformação viscoplástica;
\item[$ \varepsilonl $] 	Tensor de deformações (notação de Voigt);
\item[$ \varepsilonl_I $] Tensor de deformações iniciais (notação de Voigt);
\item[$\varepsilonl^e$] Deformação elástica (notação de Voigt);
\item[$\varepsilonl^p$] Deformação plástica (notação de Voigt);
\item[$\varepsilonl^{vp}$] Deformação viscoplástica (notação de Voigt);

\item[$\bar{\varepsilon}^{vp}$] Magnitude da deformação viscoplástica acumulada;
\item[$ \varepsilon_I $] Deformação a partir da qual tem-se a coesão residual $c_0$;
\item[$ \varepsilon_a $] Deformação axial em um ensaio triaxial;
\item[$ \varepsilon_R $] Tolerância para o resíduo;
\item[$ \varepsilon_u $] Tolerância para o incremento de deslocamentos;

\item[$ \eta $] 		Constante de viscosidade dinâmica;
\item[$\eta_1,\eta_2,\eta_3$] 	Direções principais;
\item[$ \theta $] 		Ângulo de Lode;
\item[$\lambda$] 	Magnitude da deformação plástica ou viscoplástica;
\item[$\lambda^e$] 	Coeficiente de Lamè;
\item[$\mu^e$] 	Coeficiente de Lamè;
\item[$ \nu $] 			Coeficiente de Poisson;
\item[$ \xi_{H} $] 				Coordenada do espaço de Haigh-Westergaard;
\item[$\underline \xi$] 				Coordenada naturais do elemento;
\item[$\xil_{i_p}$] 				Coordenada natural do ponto de integração $i_p$;


\item[$ \phi $] 		Ângulo de atrito do maciço;
\item[$ \rho $] 		Densidade do domínio;
\item[$ \rho_{H} $] 		Coordenada do espaço de Haigh-Westergaard;

\item[$ \sigma_{oct} $] 	Tensão normal octaédrica;
\item[$ \sigma_{v} $] 	Tensão vertical no interior do maciço;
\item[$ \sigma_{\theta \theta} $] Tensão no maciço na direção ortorradial;
\item[$ \sigma_{zz} $] 	Tensão no maciço ortogonal ao plano da seção do túnel;
\item[$ \sigma_{rr} $] 	Tensão no maciço na direção radial;
\item[$ \sigmall $] 	Tensor de tensões;
\item[$ \sigmall_I $] 	Tensor de tensões inical;
\item[$ \sigmal $] 	Tensor de tensões (notação de Voigt);
\item[$ \sigmal_I $] 	Tensor de tensões iniciais (notação de Voigt);
\item[$ \sigmal^{trial} $] 	Preditor elástico (notação de Voigt);
\item[$ \tau_{oct} $] 	Tensão cisalhante octaédrica;
\item[$\psi$] energia livre específica ou ângulo de dilatância;

\item[\textbf{Grego maiúsculo:}]
\item[$ \Gamma $]		Conjunto de tensões plasticamente admissíveis;
\item[$ \Theta $]		Parâmetro que controla um esquema de integração totalmente explícito ou semi-implicito no modelo viscoplástico;
\item[$ \Sigma $]		Superfície de descontinuidade no interior do domínio;
\item[$ \Phi $]		Função de sobretensão;
\item[$ \Omega $] 		Domínio contínuo;
\item[$ \Omega' $] 		Subdomínio contínuo;
\item[$ \Omega_e $] 		Domínio de um elemento;
\item[$\Omega_\xi$] 		Domínio natural de um elemento;

\item[\textbf{Numéricos, operadores e funções matemáticas:}]
\item[$ \pi $] 		Constante $\pi \approxeq 3.141592653589793$;
\item[$ \Umll $] 	Tensor de segunda ordem unitário;
\item[$ \Umll $] 	Tensor de quarta ordem unitário (notação de Voigt);
\item[$ \Uml $] 	Tensor de segunda ordem unitário (notação de Voigt);
\item[$\Umllll$] Tensor unitário de quarta ordem;
\item[det($*$)] 	Função Determinante;
\item[tr($*$)] 	Função Traço;
\item[$(*)^T$] 	Operador de Transposição;
\item[$ \divl $] 	Operador Divergente;
\item[$ \nabla $] 	Operador Gradiente;
\item[$||*||$] 	Operador Módulo;
\item[$\sum *$] 	Operador Somatório;


















