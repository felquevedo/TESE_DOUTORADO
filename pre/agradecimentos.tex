Agradeço ao Conselho Nacional de Desenvolvimento Científico Tecnológico (CNPq) pela oportunidade de estudar com bolsa durante esse doutorado, sem a qual, não seria possível minha dedicação nesse período.

Agradeço à UFRGS pela infraestrutura cedida, por possibilitar acesso à biblioteca e a artigos internacionais gratuitamente, além da oportunidade de cursar esse doutorado na área de Estruturas com destacados colegas e professores.

Agradeço a todo o corpo docente e administrativo do Programa de Pós-Graduação em Engenharia Civil (PPGEC) pela dedicação e apoio durante esses anos.

Agradeço imensamente aos meus orientadores, Denise Bernaud e Samir Maghous. Obrigado por me acolherem como orientando, me apresentarem sobre o tema de túneis, por compartilharem seus conhecimentos, pelo apoio e fé que tiveram comigo durante o desenvolvimento deste trabalho.

Agradeço ao professor Inácio Benvegnu Morsch pelo apoio, ensinamentos e os bons conselhos desses anos todos. Obrigado também pelos excelentes momentos de descontração e churrascos com a turma.

Agradeço aos professores do PPGEC e do CEMACOM que além de me acolherem como aluno me ensinaram muito durante o curso.

Agradeço também aos meus professores de graduação que muito me ensinaram sobre estruturas. Professor Luiz Alberto Segovia, Roberto Rios, Acir Loredo de Souza e Américo Campos Filho.

Agradeço aos colegas Augusto Bopsin Borges e Mateus Forcelini por compartilharem os seus conhecimentos de geotecnia comigo e, juntamente com o Rodrigo Pereira, por me ajudarem a fazer o modelo Latex do qual essa tese é feita.

Agradeço aos demais colegas que me foram ótimas companhias durante esses anos todos e me enchem de saudades e boas lembranças. Obrigado Bárbara Chagas, Barbara Sanches, Bianca Funk, Bibiana Rossato, Bruna Spricigo, Cassiele Birk, Cássio Barros de Aguiar, Dani Airão, Daniel Matos, Eduardo Braun, Eduardo Gibbon Rosa, Eduardo Titello, Gabriela Bianchin (Gabi), Giancarlo Machado, Graciane Azevedo, Guilherme Fornel, Gustavo Ribeiro, Igor Souza Hoffman, Jose Rafael Yepez Aguirre (Chepel), Kellyn Pufall, Livio Pires, Lúcia Sangali, Marcela Palhares Miranda, Matheus Benincá, Matheus Wanglon, Miguel Aguirre (Miguelito), Monique Wesz Vogado, Paulo Baumbach, Raiza Guimarães, Rodrigo Benites Mendes, Rosangel Rojas Aguero (Rosi), Mateus Tonin, Paulo Baumbach, Samuel Bandeira, Sandro Pieta Troian, Tenison Freire..e muitos outros que não me lembro.

Agradeço também aos demais orientandos da Denise que sempre compartilharam um pouco de suas dissertações e teses comigo. Alex La Flor,  Betina Jensen, Caio Cesar Cardoso, Evandro Pandia Cayro, Tiago Wagner Dada e Wallace Ferrão. 

Agradeço ao suporte ténico da Engineering Simulation and Scientific Software (ESSS) pela assistência técnica ferente às dúvidas relacionadas à utilização e customização do \textit{software} ANSYS.

Agradeço ao pessoal da Estádio 3, meus ex-companheiros de trabalho, não só pelo que me ensinaram, mas também pelo apoio na minha decisão de seguir esse período acadêmico na minha vida.

Agradeço aos meus amigos Rubens Salabarry, que mesmo distante esteve sempre presente e me ajudou na revisão do texto. Agradeço também a companhia da minha amiga Fernanda Ribas Tweedy que foi minha parceira de bike e de conversas filosóficas.

Agradeço a minha mãe Luzia e minha irmã Rita por estarem sempre perto de mim, confiando e incentivando os meus estudos, apesar de todas as dificuldades.



