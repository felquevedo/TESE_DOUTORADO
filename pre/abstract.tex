The design and structural verification of deep tunnels involves several geotechnical parameters as well as the need to predict convergence (closure) and intuit the stability of the tunnel section. However, the field of deformations and stresses around the cavity depends on several interrelated factors, such as depth, section geometry, in situ stress anisotropy, rockmass heterogeneity, constructive process and lining interaction. Added to this complexity the rheology of the rockmass is a crucial factor. The behavior of the rockmass is composed of an instantaneous (elastic or elastoplastic) and time dependent (viscous) portion. The main objective of this thesis is to formulate, program and validate, in the context of finite elements method, a general elastoplastic-viscoplastic constitutive model capable of reproducing cases of reality and allowing studies on this behavior. For the instantenous part will be adopted elastoplastic models fo Drucker-Prager and Mohr-Coulomb (as well as their independent pressure simplifications, von-Mises and Tresca, respectively) with multilinear law hardening/softening governed by the cohesive internal variable. The time dependent behavior will be studied through a perfect viscoplastic model according to Perzyna’s overstress theory with the same surfaces of elastoplasticity. The plane deformation state, axisimmetry and three-dimensional states will be programmed in Fortran 90. The process of excavation and placement of the lining is simulated through the method of activation and deactivation of tinite elements.