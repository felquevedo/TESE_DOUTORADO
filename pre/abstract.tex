The design and structural verification of deep tunnels involves several geotechnical parameters as well as the need to predict the convergence (closure) and intuit the stability of the tunnel cross section. However, the field of deformations and stresses around the cavity depends on several interrelated factors, such as depth, cross section geometry, anisotropy of stresses \textit{in situ}, rockmass heterogeneity, construction process and lining interaction. Added to this complexity, the rheology of the rockmassif is a crucial factor. A common aspect in the problem of tunnels is the long-term behavior. The main objective of this thesis was to formulate, program and validate, in the context of the finite elements method in plane, axisymmetry and three-dimensional state strain, a coupled constitutive model capable of simulating instantaneous strains (elastic or elastoplastic) together with long-term (viscous) behavior. For the instantaneous behavior, an elastoplastic Drucker-Prager model with hardening/softening law governed by the cohesive internal variable was adopted. In the time dependent behavior, a viscoplastic model was used according to the Perzyna overstress theory with the same elastoplasticity surface, however perfect. Therefore, a numerical integration scheme is presented to deal, in general, with the behavior coupled together with its internal variables. The model was implemented in software ANSYS 2021R1 using its programmable feature Usermat. The process of excavating and placing the lining is simulated by activating and deactivating of the finite elements. Verifications and validations with analytical and numerical solutions showed that the developed model worked properly. Differences in the order of 10\% to 30\% were found when comparing this coupled model with the viscoplastic-only model. The influence of this model on an elliptical cross section was also studied, demonstrating its importance when plasticization occurs in the rockmass.