% %%%%%%%%%%%%%%%%%%%%%%%%%%%%%%%%%%%%%%%%%%%%%%%%%%%%%%%%%%%%%%%%%%%%%%%%%%%%%%%%%
%
%	PACOTES_MACROS.tex 
%  --------------------
%
%	Este arquivo é lido no preâmbulo do main.tex e portanto, você pode inserir nele:
%
%		- pacotes que não estão na classe ppgec.cls (veja se a classe já não possui)
%		- suas próprias macros
%		- qualquer outro comando de preâmbulo
%
% %%%%%%%%%%%%%%%%%%%%%%%%%%%%%%%%%%%%%%%%%%%%%%%%%%%%%%%%%%%%%%%%%%%%%%%%%%%%%%%%%
%
%	Pacotes
%	-------
%
\usepackage{float}	% permite posicionar a figura entre parágrafos usando H
%
%
% %%%%%%%%%%%%%%%%%%%%%%%%%%%%%%%%%%%%%%%%%%%%%%%%%%%%%%%%%%%%%%%%%%%%%%%%%%%%%%%%%
%
%	Macros e configurações de pacotes
%	---------------------------------
%
% Imprimir anexo pdf: \imprimiranexopdf{nome do arquivo}{título do anexo}{primeirapagina}{proximaspaginas}
\newcommand{\imprimiranexopdf}[4]{
	\includepdf[pages=#3,pagecommand=\chapter{#2},scale=0.92,offset=0 -95]{#1}
	\includepdf[pages=#4,pagecommand={}]{#1}
}
%
%
% %%%%%%%%%%%%%%%%%%%%%%%%%%%%%%%%%%%%%%%%%%%%%%%%%%%%%%%%%%%%%%%%%%%%%%%%%%%%%%%%%
%
%	Outros comandos de preâmbulo
%	----------------------------
\newcommand{\gammal}{\underline \gamma}
\newcommand{\fl}{\underline f}
\newcommand{\Tl}{\underline T}
\newcommand{\euml}{\underline{e}_1}
\newcommand{\edoisl}{\underline{e}_2}
\newcommand{\etresl}{\underline{e}_3}
\newcommand{\xl}{\underline x}
\newcommand{\sigmall}{\underline{\underline \sigma}}
\newcommand{\nl}{\underline n}
\newcommand{\divl}{\underline \nabla \cdot}
\newcommand{\ul}{\underline u}
\newcommand{\Fll}{\underline{\underline F}}
\newcommand{\Umll}{\underline{\underline 1}}
\newcommand{\greenll}{\underline{\underline e}}
\newcommand{\varepsilonll}{\underline{\underline \varepsilon}}
\newcommand{\Dllll}{\underline{\underline{\underline{\underline D}}}}
\newcommand{\Umllll}{\underline{\underline{\underline{\underline 1}}}}
\newcommand{\alphal}{\underline \alpha}
\newcommand{\ql}{\underline q}
\newcommand{\sll}{\underline{\underline s}}
\newcommand{\dgdsll}{\underline{\underline {g_{_\sigma}}}}
\newcommand{\gllum}{\underline{\underline {g_{_1}}}}
\newcommand{\glldois}{\underline{\underline {g_{_2}}}}
\newcommand{\glltres}{\underline{\underline {g_{_3}}}}
\newcommand{\hl}{\underline h}
\newcommand{\dfdsll}{\underline{\underline {f_{_\sigma}}}}
\newcommand{\dfdql}{\underline{f_{_q}}}
\newcommand{\dul}{\underline {\delta u}}
\newcommand{\dvarepsilonll}{\underline{\underline {\delta \varepsilon}}}

\newcommand{\varepsilonl}{\underline{\varepsilon}}
\newcommand{\Dll}{\underline{\underline D}}
\newcommand{\sigmal}{\underline \sigma}
\newcommand{\dgdsl}{\underline{g_{_\sigma}}}
\newcommand{\dfdsl}{\underline{f_{_\sigma}}}
\newcommand{\Uml}{\underline{1}}
\newcommand{\dvarepsilonl}{\underline{\delta \varepsilon}}
\newcommand{\Nll}{\underline{\underline N}}
\newcommand{\Bll}{\underline{\underline B}}
\newcommand{\Fl}{\underline{F}}
\newcommand{\Kll}{\underline{\underline K}}
\newcommand{\Jll}{\underline{\underline J}}
\newcommand{\xil}{\underline{\xi}}
\newcommand{\Rl}{\underline{R}}
\newcommand{\Lll}{\underline{\underline L}}
\newcommand{\Mll}{\underline{\underline M}}
\newcommand{\al}{\underline{a}}
\newcommand{\bl}{\underline{b}}
\newcommand{\zerol}{\underline{0}}
\newcommand{\All}{\underline{\underline A}}
%
%