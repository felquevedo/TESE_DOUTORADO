\chapter{Conclusões}\label{Conclusoes}


O campo de deformações e tensões ao redor da cavidade de um túnel profundo depende de diversos fatores inter-relacionados, tais como, a profundidade, a geometria da seção, a anisotropia das tensões \textit{in situ}, a heterogeneidade do maciço, o processo construtivo e a interação com o revestimento. Somado a essa complexidade um dos maiores desafios é modelar o comportamento reológico do maciço. Os túneis podem apresentar tanto deformações instantâneas quanto diferidas ao longo do tempo. Além disso, o problema da interação entre o maciço e o revestimento é um problema essencialmente tridimensional e fortemente acoplado. Apenas casos em que o túnel profundo admite simetria cilíndrica (por exemplo, túneis de seção circular, meios isotrópicos e estado de tensões homogêneas) o cálculo exato por via numérica em axissimetria fica de um tamanho razoável, rápido e de fácil utilização. Contudo, em outras situações, é necessário ter disponível um modelo tridimensional.

Em vista disso, o objetivo principal desse trabalho foi justamente o desenvolvimento de uma solução para o problema constitutivo elastoplástico-viscoplástico que pudesse abordar problemas de túneis de uma forma geral em estado plano de deformações, em axissimetria e tridimensionais. Portanto, foi formulado e implementado em um \textit{software} de elementos finitos (ANSYS 2021R1) uma relação constitutiva capaz de abordar, dentro dessa generalidade, as deformações instantâneas (sejam elásticas ou elastoplásticas) induzidas pelo processo construtivo com as deformações viscosas que ocorrem ao longo do tempo. Os esquemas de integração numérica que resolvem esse problema constitutivo estão descritos na \autoref{algoritmo de atualização das tensões e variáveis internas} e se mostraram adequados para a resolução da lei constitutiva elastoplástica-viscoplástica. Verificações com soluções analíticas e outro \textit{software} (GEOMEC91) demonstraram que esse modelo, bastante geral, consegue reproduzir cada comportamento em separado (elástico, viscoplástico e elastoplástico). O modelo constitutivo também conseguiu reproduzir com bastante precisão a solução analítica de \citeonline{Piepi1995} para um túnel em um meio elastoplástico-viscoplástico com critério de Tresca.

O modelo também foi capaz de simular o fenômeno de amolecimento/endurecimento característico dos maciços frágeis e dúcteis. Em elastoplasticidade, adotando as mesmas propriedades de \citeonline{Piepi1995}, uma redução na coesão pela metade devido ao amolecimento conduziu a uma convergência 37\% maior. E para o modelo elastoplástico-viscoplástico, o mesmo amolecimento representou uma convergência 28\% maior, no final da construção do túnel, e 24\% maior no longo prazo.

Foram também feitas algumas análises paramétricas para mostrar as diferenças do modelo acoplado em relação ao modelo viscoplástico comumente usado para análises de longo prazo. Utilizando novamente as propriedades de \citeonline{Piepi1995}, sem a presença do revestimento, foram encontradas diferenças significativas de 30\%, no final da construção do túnel, e 50\%, no longo prazo. Com a presença do revestimento a diferença foi de 17\% a 23\% para uma distância não suportada de $d0=0$ e de 26\% a 30\% para uma distância não suportada $d0=4L_p$. Essa diferença ocorre pois o modelo constitutivo acoplado considera a plastificação do maciço nas deformações instantâneas, enquanto que o modelo viscoplástico considera essas deformações como elásticas.

Uma vez que o modelo constitutivo foi generalizado para o caso tridimensional, foi feita uma análise paramétrica considerando a variação da razão de aspecto de uma seção transversal elíptica. Como a seção é elíptica as convergências são medidas ao longo do maior e menor raio da seção. De uma forma geral, quando se aumenta a razão de aspecto os deslocamentos aumentam. Sem revestimento, uma razão de aspecto de 2, levou a maiores convergências no modelo elastoplástico-viscoplástico na direção $x$ (direção que se tem as maiores plastificações) em relação ao modelo viscoplástico. Foi encontrado uma convergência 2,05 vezes maior no curto prazo. Na direção $y$ (onde a plastificação não é tão intensa) encontrou-se uma convergência 1,22 vezes maior do que o modelo viscoplástico no curto prazo. No longo prazo, para ambas as direções, os valores ficaram equivalentes dado o nível de plastificação próximo em ambos os modelos nessa direção. Contudo, foi também constatado que o revestimento pode bloquear as deformações plásticas instantâneas, fazendo com que os modelos não apresentem diferenças tão significativas.

Frente a essas análises, foi constatada a relevância do modelo constitutivo acoplado frente ao modelo viscoplástico e é importante que esse modelo seja estudado com mais profundidade e ampliado. Portanto, como sugestões para serem adicionadas em trabalhos futuros, pode-se citar:

\begin{alineas}
	
	\item fazer mais validações com estudos experimentais;
	
	\item fazer mais estudos paramétricos para compreender melhor o comportamento desse modelo constitutivo acoplado, como, por exemplo, considerando a anisotropia das tensões inicias;
	
	\item adicionar outras superfícies de plasticidade e viscoplasticidade além da superfície de Drucker-Prager. Apesar dessa superfície ser bastante geral e incluir a dependência da pressão hidrostática, há superfícies especializadas em captar aspectos próprios da geomecânica de túneis;
	
	\item adicionar uma lei de amolecimento e endurecimento que considere a variação do ângulo de atrito durante a deformação plástica;
	
	\item aplicar esse modelo constitutivo em domínios mais complexos como o caso de túneis gêmeos e encontro de galerias;
	
	\item fazer um estudo considerando revestimentos que possuem comportamento diferido, como o caso do concreto que sofre fluência e retração;
	
	\item construir curvas de convergências para o modelo elastoplástico-viscoplástico e propor ajustes no método NIM;
	
	\item generalizar a solução para considerar grandes deformações.	
	
\end{alineas}