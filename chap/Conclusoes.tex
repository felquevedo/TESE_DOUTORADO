\chapter{Conclusões}\label{Conclusoes}

Durante o desenvolvimento dessa tese foi formulado e implementado em um \textit{software} de elementos finitos (ANSYS) uma relação constitutiva capaz de abordar as deformações instantâneas (sejam elásticas ou elastoplásticas) induzidas pelo processo construtivo juntamente com as deformações viscosas que ocorrem ao longo do tempo. 

O ponto central desse trabalho foi justamente o desenvolvimento da relação constitutiva elastoplástica-viscoplástica generalizada para abordar casos em estado plano de deformações, axissimetria e tridimensionais. Os esquemas de integração numérica descritos no Capitulo 6.5, para resolver o problema constitutivo se mostraram eficientes na busca da solução desse modelo.

Verificações com soluções analíticas e outro \textit{software} (GEOMEC91) demonstraram que esse modelo, bastante geral, consegue reproduzir cada comportamento em separado (elástico, viscoplástico e elastoplástico) e também abordar o aspecto do amolecimento/endurecimento característico do maciço.