% ----------------------------------------------------------
\chapter{Comportamento Mecânico de Túneis}
% ----------------------------------------------------------

\section{Influência da escavação e o conceito de convergência da seção}

De uma forma geral, do ponto de vista do maciço, a escavação de um túnel nada mais é do que uma perturbação no seu estado natural de equilíbrio devido à remoção de parte do maciço. Essa perturbação induzirá o maciço a uma nova configuração de equilíbrio que mobilizará tensões tangenciais desviando, dessa forma, a direção das tensões principais no entorno da escavação. Através desse arqueamento das tensões, o próprio maciço participa da sustentação da cavidade. Esse arqueamento pode ser decomposto em dois arcos longitudinais (contidos nos planos horizontal e vertical) e um arco transversal (no plano perpendicular ao eixo do túnel), tal como mostrado na Figura 4.1.










\section{Mecanismos de ruptura em túneis profundos}

\section{Influência da reologia do maciço}

\subsection{Comportamento instantâneo}

\subsection{Comportamento diferido no tempo}

\subsection{Alguns estudos considerando leis elastoplásticas e viscoplásticas}

\section{Influência da forma da seção}

\section{Influência da profundidade do túnel}

\section{Influência da proximidade da superfície}

\section{Influência do revestimento e parâmetros adimensionais}

\section{Método convergência-confinamento}




